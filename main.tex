\documentclass[a4paper,11pt]{article}

\usepackage[francais]{babel}

\usepackage{matis_proposal}

\title{Title}
\date{}

\begin{document}

    \maketitle

    \section*{Contexte}
    \noindent
    \par
    L'équipe MATIS du Laboratoire LaSTIG de l'Institut National de l'information géographique et forestière (IGN) mène depuis plusieurs années

    \section*{Objectifs du stage}


    %volume de données
    Travail à réaliser :
    \begin{itemize}
    \item
    \end{itemize}



    \subsection*{Environnement technique}
    \begin{itemize}
        \item[OS:] Linux (Ubuntu);
        \item[Language:] Python:
        \begin{itemize}
            \item scikit-learn,
            \item opencv,
            \item scikit-image.
        \end{itemize}
    \end{itemize}

    \subsection*{Compétences}
    Vision par ordinateur, apprentissage statistique, programmation informatique.


    \subsection*{Durée \& Rémunération}
    \begin{itemize}
        \item $4/5$ mois \`A partir de février 2018,
        \item $554.40$ \euro/mois.
    \end{itemize}

    \subsection*{Candidature}
    Un \textit{C.V.} et une lettre de motivation.

    \section*{Contacts}
    IGN est commanditaire du stage. L'encadrement scientifique se assurer par:
    \begin{itemize}
        \item[--] \textsc{Oussama Ennafii \& Clément Mallet:}
        \begin{itemize}
            \item[\underline{Adresse:}] Equipe MATIS, laboratoire LaSTIG, Institut National de l'Information Géographique et Forestière (IGN);
            \item[\underline{Téléphone}:] (+33) 1 43 98 84 36;
            \item[\underline{Courriel}:] \texttt{prenom.nom@ign.fr};
            \item[\underline{Web}:] \url{http://recherche.ign.fr/labos/matis}.
        \end{itemize}
    \end{itemize}
\end{document}
