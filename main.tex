\documentclass[a4paper,11pt]{article}

\usepackage[francais]{babel}

\usepackage{matis_proposal}

\title{\Large \textsc{Clustering de maquettes de bâtiments pour la qualification de méthode de reconstruction}}
\date{}

\begin{document}

    \maketitle

    \section*{Contexte}

    L'équipe MATIS du Laboratoire LaSTIG de l'Institut National de l'information géographique et forestière (IGN) mène depuis plusieurs années des projets de reconstruction automatique de modèles urbain. En effet, dans un cadre opérationnel, la reconstruction urbaine requiert l'intervention d'un opérateur pour saisir des formes géométriques (points, segments \dots), dans un cadre intéractif, ou de corriger les modèles en sortie des méthodes automatiques. Afin de faciliter l'intervention fastidieuse des opérateurs, on cherche à mettre en place une méthode de qualification automatique de de maquettes de bâtiments à partir des données intrinsèques. La sortie d'un tel algorithme permettra de distinguer les erreurs selon une taxonomy établie en avance.\\

    Le problème a été donc formuler de façon à determiner, grâce à des méthodes d'apprentissage supervisé, les erreurs de chaque bâtiments. On prend en entrée le modèle de bâtiment à qualifier, le Modèle Numérique du Terrain (MNS) --- qui a servit à la production de la maquette --- et l'orthoimage correspondants. On extrait, ainsi, des attributs géométriques, altimétriques et radiométriques afin de former un vecteur d'attributs. Les bâtiments sont classés ainsi selon leur erreurs détectées.\\

    \section*{Objectifs du stage}

    Le but du stage est d'investiguer l'apport du Clustering --- et des approches non supervisées --- pour classer les maquettes en entrée.\\

    Travail à réaliser:
    \begin{itemize}
        \item[--] En reposant sur les attributs déjà présents, évaluer les méthodes de Clustering pour classer les bâtiments;
        \item[--] Chercher les bons attributs qui pourront améliorer le Clustering;
        \item[--] Interpréter les clusters en sortie en comparant avec la taxonomy préétablie.
    \end{itemize}

    \subsection*{Environnement technique}
    \begin{itemize}
        \item[OS:] Linux (Ubuntu);
        \item[Language:] Python:
        \begin{itemize}
            \item[--] scikit-learn,
            \item[--] opencv,
            \item[--] scikit-image.
        \end{itemize}
    \end{itemize}

    \subsection*{Compétences}
    \begin{itemize}
      \item[--] Vision par ordinateur;
      \item[--] Apprentissage statistique;
      \item[--] Programmation informatique.
    \end{itemize}

    \subsection*{Durée \& Rémunération}
    \begin{itemize}
        \item[--] $4/5$ mois \`A partir de février 2018,
        \item[--] $554.40$ \euro/mois.
    \end{itemize}

    \subsection*{Candidature}
    \begin{itemize}
        \item[--] un \textit{C.V.};
        \item[--] une lettre de motivation.
    \end{itemize}

    \section*{Contacts}
    IGN est commanditaire du stage. L'encadrement scientifique sera assurer par:
    \begin{itemize}
        \item[--] \textsc{Oussama Ennafii \& Clément Mallet:}
        \begin{itemize}
            \item[\underline{Adresse:}] Equipe MATIS, laboratoire LaSTIG, Institut National de l'Information Géographique et Forestière (IGN), 73 avenue de Paris, 94165 Saint Mandé;
            \item[\underline{Téléphone}:] (+33) 1 43 98 84 36;
            \item[\underline{Courriel}:] \texttt{<prenom>.<nom>@ign.fr};
            \item[\underline{Web}:] \url{http://recherche.ign.fr/labos/matis}.
        \end{itemize}
    \end{itemize}
\end{document}
