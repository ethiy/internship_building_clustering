\documentclass[a4paper,11pt]{article}
\usepackage{indentfirst}
\usepackage[pdftex]{graphicx}
\usepackage[T1]{fontenc}
\usepackage[utf8]{inputenc}
\usepackage{color}
\usepackage[francais]{babel}
\usepackage{ulem,geometry,multicol, multirow}
\usepackage{ae,url}
\DeclareGraphicsExtensions{.pdf,.jpeg,.png}
\normalem
\geometry{margin={70pt,70pt}}

%%%% debut macro %%%%
\makeatletter    % <=== in a .sty file delete this

\newcommand{\unnumberedcaption}%
	{\@dblarg{\@unnumberedcaption\@captype}}

\newcommand{\@unnumberedcaption}{}% undefined yet
\long\def\@unnumberedcaption#1[#2]#3{\par
  \addcontentsline{\csname ext@#1\endcsname}{#1}{%
    % orig: \protect\numberline{\csname the#1\endcsname}%
    %{\ignorespaces #2}
    \protect\numberline{}{\ignorespaces #2}%
    }%
  \begingroup
    \@parboxrestore
    \normalsize
    % orig: \@makecaption{\csname fnum@#1\endcsname}%
    %{\ignorespaces #3}\par
    \@makeunnumberedcaption{\ignorespaces #3}\par
  \endgroup}

% redefine \@makeunnumberedcaption (like \@makecaption)
% for your own layout
\newcommand{\@makeunnumberedcaption}[1]{%
  \vskip\abovecaptionskip
  \sbox\@tempboxa{#1}%
  \ifdim \wd\@tempboxa >\hsize
    #1\par
  \else
    \global \@minipagefalse
    \hbox to\hsize{\hfil\box\@tempboxa\hfil}%
  \fi
  \vskip\belowcaptionskip}

% for LaTeX 2.09 compatibility, define \above/belowcaptionskip:
\@ifundefined{abovecaptionskip}{%
  \newlength{\abovecaptionskip}%
  \setlength{\abovecaptionskip}{10pt}%
}{}
\@ifundefined{belowcaptionskip}{%
  \newlength{\belowcaptionskip}%
  \setlength{\belowcaptionskip}{0pt}%
}{}

\makeatother    % <=== in a .sty file delete this
%%%% fin macro %%%%
\newcommand{\LOIC}[1]{\textcolor{red}{#1}}
\begin{document}
\hspace{-1.0cm}
\begin{center}
\begin{tabular}{ccc}
\includegraphics[height=0.08\textheight]{images/logo_ign.jpg} & \includegraphics[height=0.08\textheight]{images/matis.jpg} & \includegraphics[height=0.06\textheight]{images/logo_paris_est}\\
\end{tabular}
\end{center}
\vspace{1ex}
\begin{center}
\Large{
Title
}
\end{center}

\vspace{0.5cm}
\begin{center}
IGN - Laboratoire LaSTIG - Equipe MATIS\\73 avenue de Paris 94165 Saint Mandé.
\end{center}


\section*{Contexte}
\noindent
\par
L'équipe MATIS du Laboratoire LaSTIG de l'Institut National de l'information géographique et forestière (IGN) mène depuis plusieurs années

\section*{Objectifs du stage}


%volume de données
Travail à réaliser :
\begin{itemize}
\item
\end{itemize}



\subsection*{Environnement technique}
Python/C++ comme langage de programmation.

\subsection*{Compétences}
Traitement d'images, apprentissage profond, télédétection, séries temporelles, programmation informatique.


\subsection*{Durée \& Rémunération}
4-5 mois -- \`A partir de février 2018 -- 554.40 euros nets / mois.

\subsection*{Candidature}
1 CV et une lettre de motivation.

\section*{Contacts}
IGN est commanditaire du stage. L'encadrement scientifique de ce stage

\noindent
\textbf{Ous \& Clem \& Arnaud}\\
\indent \underline{Adresse}: Equipe MATIS/laboratoire LaSTIG - Institut National de l'Information Géographique et Forestière (IGN),\\
\indent 73 avenue de Paris 94165 Saint Mandé\\
\indent \underline{Téléphone}: (+33) 1 43 98 84 36\\
\indent \underline{Courriel}: \texttt{prenom.nom@ign.fr}\\
\indent \underline{Web}: \url{http://recherche.ign.fr/labos/matis}\\

\end{document}
